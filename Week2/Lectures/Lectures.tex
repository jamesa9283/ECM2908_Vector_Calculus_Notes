\documentclass{article}

% Packages
\usepackage{fullpage}
\usepackage{multicol}
\usepackage{amsmath}
\usepackage{mathtools}
\usepackage{bm}
\usepackage{tikz}
\usetikzlibrary{shapes.geometric, positioning}

% Macros
\newcommand{\R}{\mathbb{R}}
\newcommand{\N}{\mathbb{N}}
\newcommand{\Q}{\mathbb{Q}}
\renewcommand{\vec}[1]{\underline{\textbf{#1}}}
\newcommand{\veci}{\bm{\hat{\imath}}}
\newcommand{\vecj}{\bm{\hat{\jmath}}}
\newcommand{\veck}{\bm{\hat{k}}}
\newcommand{\e}{\varepsilon}
\newcommand{\de}{\delta}
\newcommand{\kd}{\delta_{i, j}}
\newcommand{\at}{\e_{i, j, k}}
\newcommand{\nab}{\underline{\nabla}}
\newcommand{\grad}{{\nab}\, f}
\newcommand{\pd}[2]{\frac{\partial #1}{\partial #2}}
\renewcommand{\div}[1]{\nab \cdot #1}
\newcommand{\curl}[1]{\nab \times #1}

\newtheorem{example}{Example}
\newtheorem{solution}{Solution}
\newtheorem{definition}{Definitions}

\title{Vector Calculus Week 2 - More Suffix Notation}
\author{James Arthur}

\begin{document}
\maketitle
\tableofcontents\newpage


\multicols{2}

\section{Gradient, Divergence and Curl}
\subsection{Gradient}

Assume we have a $f = f(x, y, z)$ or $f = f(x_1,x_2,x_3)$, so a scalar calued function. Then we define grad f as:
$$ \grad = \left(\pd{}{x}\veci + \pd{}{y}\vecj + \pd{}{z}\veck \right)\, f$$
We say grad of $f$ is a differential operator. So:
$$ \grad = \left(\pd{f}{x}\veci + \pd{f}{y}\vecj + \pd{f}{z}\veck \right) $$
and we can write it in suffix notation aswell:
$$ \left[ \grad \right]_i = \pd{}{x_i} \qquad i = 1, 2, 3$$

\subsection{Divergence}

Assume we have a vector field, $\vec{u} = \vec{u} (x, y, z, t)$. We define the divergence of this vector field as;
$$ \nab \cdot \vec{u} = \left(\pd{u_1}{x_1} + \pd{u_2}{x_2} + \pd{u_3}{x_3}\right)$$
Placing this in suffix notation, we get that:
$$ [\nab \cdot \vec{u}]_j = \pd{u_j}{x_j} $$

\subsection{Curl}

the curl of a vector field can be written as:
$$ \nab \times \vec{u} $$
To write this in suffix notation, we can just use the cross produce formula:
$$ [\nab \times \vec{u}]_i = \e_{ijk} \nab_j u_k $$
which then can be manipulated into:
$$ [\nab \times \vec{u}]_i = \e_{ijk} \pd{u_k}{x_j} \qquad j,k = 1, 2, 3 $$
where $i$ is a free index and $j, k$ are dummy suffixes, so $j, k = 1, 2, 3$

\section{Combinations of $\grad$, $\div(\quad)$ and $\curl(\quad)$}

If we take $\nab \cdot \grad$ where $f = (x_1, x_2, x_3, t)$. We can write the div of grad as:
\begin{align*}
  \div \grad &= \left(\pd{}{x}\veci + \pd{}{y}\vecj + \pd{}{z}\veck \right) \cdot \left(\pd{f}{x}\veci + \pd{f}{y}\vecj + \pd{f}{z}\veck \right) \\
  &= \pd{}{x_1}\pd{f}{x_1} + \pd{}{x_2}\pd{f}{x_2} + \pd{}{x_3}\pd{f}{x_3}\\
  &= \pd{^2 f}{x_1^2} + \pd{^2 f}{x_2} + \pd{^2 f}{x_3}\\
  &= \Delta f
\end{align*}

Where the $\Delta = \nab ^2$ is the laplacian. So how do we write this in suffix notation?
\begin{align*}
  \div \grad &= \nab_j[\grad]_j\\
  &= \pd{}{x_j} \pd{f}{x_j}\\
  &= \pd{^2 f}{x_j}\\
\end{align*}














\end{document}
